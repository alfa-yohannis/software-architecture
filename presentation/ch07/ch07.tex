\documentclass[aspectratio=169, table]{beamer}

\usepackage{listings}

\lstdefinestyle{RustStyle}{
	language=Java,
	morekeywords={println, Ok, async, fn, main, use, let, mut},
	basicstyle=\ttfamily\scriptsize,
	keywordstyle=\color{blue},
	commentstyle=\color{gray},
	stringstyle=\color{red},
	breaklines=true,
	showstringspaces=false,
	tabsize=2,
	captionpos=b,
	numbers=left,
	numberstyle=\tiny\color{gray},
	frame=lines,
	backgroundcolor=\color{lightgray!10},
	comment=[l]{//},
	morecomment=[s]{/*}{*/},
	commentstyle=\color{gray}\ttfamily,
	string=[s]{'}{'},
	morestring=[s]{"}{"},
	stringstyle=\color{teal}\ttfamily,
	%	showstringspaces=false
	literate=
	{\{}{{\textcolor{red}{\{}}}1
	{\}}{{\textcolor{red}{\}}}}1
	{:}{{\textcolor{red}{:}}}1
	{=}{{\textcolor{red}{=}}}1
	{.}{{\textcolor{red}{.}}}1
	{]}{{\textcolor{red}{]}}}1
	{[}{{\textcolor{red}{[}}}1
	{\#}{{\textcolor{red}{\#}}}1
	{;}{{\textcolor{red}{;}}}1
	{?}{{\textcolor{red}{?}}}1
	{!}{{\textcolor{red}{!}}}1
}

%\usepackage[beamertheme=./praditatheme]{Pradita}

\usetheme{Pradita}

\lstdefinelanguage{bash} {
	keywords={},
	basicstyle=\ttfamily\small,
	keywordstyle=\color{blue}\bfseries,
	ndkeywords={iex},
	ndkeywordstyle=\color{purple}\bfseries,
	sensitive=true,
	commentstyle=\color{gray},
	stringstyle=\color{red},
	numbers=left,
	numberstyle=\tiny\color{gray},
	breaklines=true,
	frame=lines,
	backgroundcolor=\color{lightgray!10},
	tabsize=2,
	comment=[l]{\#},
	morecomment=[s]{/*}{*/},
	commentstyle=\color{gray}\ttfamily,
	stringstyle=\color{purple}\ttfamily,
	showstringspaces=false
}

\lstdefinestyle{JavaStyle}{
language=Java,
basicstyle=\ttfamily\footnotesize,
morekeywords={String, int},
keywordstyle=\color{blue},
commentstyle=\color{gray},
stringstyle=\color{red},
breaklines=true,
showstringspaces=false,
tabsize=2,
captionpos=b,
numbers=left,
numberstyle=\tiny\color{gray},
frame=lines,
backgroundcolor=\color{lightgray!10},
comment=[l]{//},
morecomment=[s]{/*}{*/},
commentstyle=\color{gray}\ttfamily,
string=[s]{'}{'},
morestring=[s]{"}{"},
%	stringstyle=\color{teal}\ttfamily,
%	showstringspaces=false
}

\title{\Huge Peer-to-peer\\Architecture\\\vspace{10pt}}
\subtitle{IF231303-Software Architecture}
\author{Alfa Yohannis}
\begin{document}

\frame{\titlepage}

\begin{frame}[fragile]
\frametitle{Contents}
\vspace{10pt}
\begin{columns}[t]
\column{0.5\textwidth}
\tableofcontents[sections={1-4}]

\column{0.5\textwidth}
\tableofcontents[sections={5-10}]
\end{columns}
\end{frame}


\section{Introduction}

\begin{frame}{Introduction: Peer-to-Peer Architecture}
	\vspace{20pt}
	\begin{itemize}
		\item \textbf{Peer-to-peer (P2P)} is a distributed computing model.
		\item Each node (\textbf{peer}) acts as both client and server.
		\item Differs from traditional \textbf{client-server} architecture relying on a central authority.
		\item Enables direct communication and resource sharing between peers.
		\item Advantages:
		\begin{itemize}
			\item More resilient, scalable, efficient.
			\item Eliminates \textbf{single point of failure}.
			\item Enhances system redundancy.
		\end{itemize}
		\item Applications include:
		\begin{itemize}
			\item File-sharing networks
			\item Streaming services
			\item Blockchain platforms and decentralized applications (Web3)
		\end{itemize}
	\end{itemize}
\end{frame}

\section{History of Peer-to-Peer Architecture}

\begin{frame}{History of Peer-to-Peer Architecture}
	\vspace{20pt}
	\begin{itemize}
		\item Early systems: direct node communication for resource sharing.
		\item \textbf{1999: Napster} popularised P2P for music file sharing.
		\begin{itemize}
			\item Still used a central server for file indexing.
		\end{itemize}
		\item 2000s: emergence of \textbf{Gnutella}, \textbf{Freenet}, and \textbf{BitTorrent}.
		\begin{itemize}
			\item Eliminated reliance on central servers.
			\item BitTorrent introduced \textbf{tracker}-based large-scale distribution.
		\end{itemize}
		\item 2010s: \textbf{Blockchain} reinforced P2P.
		\begin{itemize}
			\item \textbf{Bitcoin (2008)}: distributed ledger system.
			\item Foundation for DApps, DeFi, smart contracts.
		\end{itemize}
		\item Today:
		\begin{itemize}
			\item Supports distributed computing, CDNs, edge computing.
			\item Core component of modern infrastructure.
		\end{itemize}
	\end{itemize}
\end{frame}


\section{Advantages of Peer-to-Peer Architecture}

\begin{frame}{Advantages of Peer-to-Peer Architecture}
	\vspace{20pt}
	\begin{columns}[t]
		
		\begin{column}{0.5\textwidth}
			\begin{enumerate}
				\item \textbf{Decentralised} \\
				No central server, no single point of failure.
				
				\item \textbf{Scalable} \\
				Grows easily as peers join, sharing resources.
				
				\item \textbf{Cost-Efficient} \\
				Uses peer resources, reducing server costs.
				
				\item \textbf{Fault-Tolerant} \\
				Node failure doesn’t disrupt the network.
			\end{enumerate}
		\end{column}
		
		\begin{column}{0.5\textwidth}
			\begin{enumerate}
				\setcounter{enumi}{4}
				\item \textbf{Load Balanced} \\
				Workload spread across many peers.
				
				\item \textbf{Fast Content Delivery} \\
				Peers upload/download simultaneously (e.g., BitTorrent).
				
				\item \textbf{Censorship-Resistant} \\
				Hard to block due to decentralisation.
				
				\item \textbf{Privacy-Friendly} \\
				Systems like \textbf{Tor} enhance anonymity.
			\end{enumerate}
		\vspace{10pt}
			Ideal for distributed computing, file sharing, \textbf{blockchain}, and \textbf{Web3}.
		\end{column}
		
	\end{columns}

\end{frame}

\section{Disadvantages of Peer-to-Peer Architecture}

\begin{frame}{Disadvantages of Peer-to-Peer Architecture}
	\begin{columns}[t]
		
		\begin{column}{0.5\textwidth}
			\begin{enumerate}
				\item \textbf{Security Risks} \\
				No central control:
				\begin{itemize}
					\item \textbf{Sybil attacks}: fake peers disrupt network.
					\item \textbf{DoS attacks}: floods with requests.
					\item \textbf{Malware spread}.
				\end{itemize}
				
				\item \textbf{Data Issues} \\
				Peers offline = unavailable/inconsistent data.
				
				\item \textbf{Inefficient Search} \\
				Flooding/random walk increases latency and traffic.
				
				\item \textbf{Legal Issues} \\
				Hard to control illegal content sharing.
			\end{enumerate}
		\end{column}
		
		\begin{column}{0.5\textwidth}
			\begin{enumerate}
				\setcounter{enumi}{4}
				
				
				
				\item \textbf{No Central Management} \\
				No admin for:
				\begin{itemize}
					\item Moderation.
					\item Consistent performance.
					\item Access control.
				\end{itemize}
				
				\item \textbf{Resource-Heavy} \\
				Each peer uses bandwidth, CPU, storage.
				
				\item \textbf{Complex Design} \\
				Needs efficient routing, security, handles dynamic peers.
			\end{enumerate}
			
			\vspace{10pt}
			Still crucial for decentralised, fault-tolerant systems.
		\end{column}
		
	\end{columns}
\end{frame}



\section{Structured P2P Networks}

\begin{frame}{Structured Peer-to-Peer Networks}
\vspace{20pt}
\begin{columns}[t]

\begin{column}{0.5\textwidth}
\textbf{Key Characteristics:}
\begin{itemize}
\item \textbf{Deterministic Search}: \\
Resources indexed systematically, ensuring efficient lookup.

\item \textbf{Scalable}: \\
Handles many peers and data with minimal search time increase.

\item \textbf{Reliable}: \\
Data redundancy prevents loss if peers leave.

\item \textbf{Efficient Routing}: \\
Fewer hops, lower latency and network traffic.
\end{itemize}
\end{column}

\begin{column}{0.5\textwidth}
\textbf{Distributed Hash Tables (DHTs):}
\begin{itemize}
\item Map data to specific peers.
\item Each node handles part of key space.
\item Key operations:
\begin{itemize}
\item \textbf{Insert (key, value)}
\item \textbf{Lookup (key)}
\item \textbf{Routing} queries efficiently
\end{itemize}
\item Forms structured overlay networks.
\item Examples: \textbf{Chord}, \textbf{Pastry}, \textbf{Kademlia}.
\end{itemize}

\vspace{10pt}
Structured P2P delivers scalable, efficient, and reliable data sharing.
\end{column}

\end{columns}
\end{frame}


\subsection{Popular Structured P2P Protocols}

\begin{frame}{Chord and Pastry Protocols}
	\vspace{20pt}
	\begin{columns}[t]
		
		\begin{column}{0.5\textwidth}
			\textbf{Chord Protocol}
			\begin{itemize}
				\item Ring-based DHT structure.
				\item Each peer assigned unique ID (hash of IP).
				\item Peers form logical ring.
				\item Uses \textbf{finger table} for efficient routing.
				\item Lookup cost: \(\mathcal{O}(\log N)\).
			\end{itemize}
			Scalable, fault-tolerant, suitable for large distributed systems.
		\end{column}
		
		\begin{column}{0.5\textwidth}
			\textbf{Pastry Protocol}
			\begin{itemize}
				\item Prefix-based routing.
				\item Peers have hashed numeric ID.
				\item Routing table used for prefix matching.
				\item Queries match prefixes progressively.
				\item Lookup cost: \(\mathcal{O}(\log N)\) hops.
			\end{itemize}
			Used in distributed file systems and content delivery networks (CDNs).
		\end{column}
		
	\end{columns}
\end{frame}



\begin{frame}{Kademlia Protocol}
	\vspace{20pt}
	\textbf{Kademlia} applies XOR distance metric to structure its network.
	\begin{itemize}
		\item Each peer has a unique ID.
		\item Distance calculated using bitwise XOR.
		\item Peers maintain detailed routing info for nearby nodes.
		\item Prefers routing through stable peers for reliability.
		\item Lookup complexity: \(\mathcal{O}(\log N)\).
	\end{itemize}
	Widely used in real-world systems like \textbf{BitTorrent DHT}.
\end{frame}

\begin{frame}{Structured P2P Networks: Pros and Cons}
	\vspace{20pt}
	\begin{columns}[t]
		
		\begin{column}{0.5\textwidth}
			\textbf{Advantages:}
			\begin{itemize}
				\item \textbf{Efficient Search:} \\
				Deterministic lookup in \(\mathcal{O}(\log N)\) hops.
				
				\item \textbf{Highly Scalable:} \\
				Supports millions of peers with minimal performance loss.
				
				\item \textbf{Low Bandwidth Use:} \\
				Optimised routing lowers traffic overhead.
				
				\item \textbf{Fault-Tolerant:} \\
				Data redundancy maintains availability when peers leave.
			\end{itemize}
		\end{column}
		
		\begin{column}{0.5\textwidth}
			\textbf{Disadvantages:}
			\begin{itemize}
				\item \textbf{High Complexity:} \\
				Requires maintaining routing tables, complex algorithms.
				
				\item \textbf{Node Change Overhead:} \\
				Joins/leaves require updating routes, redistributing data.
				
				\item \textbf{Less Flexible:} \\
				Predefined structure less adaptive to dynamic environments.
			\end{itemize}
			
			\vspace{10pt}
			Structured P2P offers efficiency but at the cost of complexity and flexibility.
		\end{column}
		
	\end{columns}
\end{frame}

\section{Unstructured P2P Networks}

\begin{frame}{Unstructured P2P Networks}
	\vspace{20pt}
	\begin{columns}[t]
		
		\begin{column}{0.6\textwidth}
			\textbf{Key Features:}
			\begin{itemize}
				\item \textbf{Random Connections:} \\
				Peers connect freely, no fixed structure.
				
				\item \textbf{Decentralised Storage:} \\
				Data spread across peers, no central mapping.
				
				\item \textbf{Failure Resilience:} \\
				Node failure has little network impact.
			\end{itemize}
			
			\textbf{Search: Flooding}
			\begin{itemize}
				\item Broadcasts to all connected peers.
				\item Stops when data found or TTL limit reached.
				\item \textbf{Reliable} but \textbf{High bandwidth, inefficient.}
			\end{itemize}
		\end{column}
		
		\begin{column}{0.4\textwidth}
			\textbf{Search: Random Walk}
			\begin{itemize}
				\item Requests sent to random peers.
				\item Fewer messages, lighter load.
				\item \textbf{+ Lower traffic.}
				\item \textbf{- Slower, less reliable.}
			\end{itemize}
			
			\vspace{10pt}
			\textbf{Summary:}
			\begin{itemize}
				\item Easy peer join/leave.
				\item Flexible but inefficient search.
				\item Good for dynamic, large-scale systems.
			\end{itemize}
		\end{column}
		
	\end{columns}
\end{frame}

\begin{frame}{Unstructured P2P: Pros and Cons}
	\vspace{20pt}
	\begin{columns}[t]
		
		\begin{column}{0.6\textwidth}
			\textbf{Advantages:}
			\begin{itemize}
				\item \textbf{Failure Resilience:} \\
				No single point of failure; network keeps running despite node failures.
				
				\item \textbf{Easy Node Participation:} \\
				Peers join freely without complex setup.
				
				\item \textbf{Censorship Resistance:} \\
				Decentralised, harder to block or control.
				
				\item \textbf{Effective for Small Scale:} \\
				Search still efficient in smaller networks.
			\end{itemize}
		\end{column}
		
		\begin{column}{0.4\textwidth}
			\textbf{Disadvantages:}
			\begin{itemize}
				\item \textbf{Inefficient Search:} \\
				No structured index; flooding increases network load.
				
				\item \textbf{High Latency at Scale:} \\
				Larger networks slow down rare data searches.
				
				\item \textbf{Excess Bandwidth Use:} \\
				Flooding generates heavy, unnecessary traffic.
			\end{itemize}
		\end{column}
		
	\end{columns}
\end{frame}

\begin{frame}{Unstructured P2P: Resource Discovery}
	\vspace{20pt}
	\begin{columns}[t]
		
		\begin{column}{0.5\textwidth}
			\textbf{Flooding:}
			\begin{itemize}
				\item Search request sent to all connected peers.
				\item Peers forward request until:
				\begin{itemize}
					\item Resource is found.
					\item TTL (Time-To-Live) limit reached.
				\end{itemize}
				\item \textbf{Pros:} High reliability.
				\item \textbf{Cons:} High bandwidth usage, inefficient at scale.
			\end{itemize}
		\end{column}
		
		\begin{column}{0.5\textwidth}
			\textbf{Random Walk:}
			\begin{itemize}
				\item Request forwarded randomly to selected peers.
				\item Reduces traffic vs flooding.
				\item \textbf{Pros:} Lowers network load.
				\item \textbf{Cons:} May take longer to locate rare resources.
			\end{itemize}
			
			\vspace{10pt}
			\textbf{Summary:} \\
			Unstructured P2P favours flexibility, but search methods trade off between efficiency and reliability.
		\end{column}
		
	\end{columns}
\end{frame}


\begin{frame}{Examples of Unstructured P2P Networks}
	\vspace{20pt}
	\begin{columns}[t]
		
		\begin{column}{0.5\textwidth}
			\textbf{Gnutella:}
			\begin{itemize}
				\item Early file-sharing P2P network.
				\item Each peer connects to multiple neighbours.
				\item Uses \textbf{flooding} to search:
				\begin{itemize}
					\item Query sent to all connected peers.
					\item Peer replies if it has requested file.
				\end{itemize}
				\item Fully decentralised, no central directory.
			\end{itemize}
		\end{column}
		
		\begin{column}{0.5\textwidth}
			\textbf{Freenet:}
			\begin{itemize}
				\item Focuses on privacy, censorship resistance.
				\item Data is encrypted and distributed.
				\item Peers cache frequently accessed data.
				\item Users upload/download anonymously.
			\end{itemize}
			
			\vspace{10pt}
			\textbf{Summary:} \\
			Unstructured P2P excels in decentralisation, privacy, and fault tolerance, but faces search inefficiency at large scale.
		\end{column}
		
	\end{columns}
\end{frame}

\begin{frame}{Hybrid Peer-to-Peer Networks}
	\vspace{20pt}
	\begin{columns}[t]
		
		\begin{column}{0.5\textwidth}
			\textbf{Definition:} \\
			Combines centralised and decentralised aspects. Central server assists certain tasks, but peers communicate directly.
			
			\vspace{10pt}
			\textbf{Key Features:}
			\begin{itemize}
				\item \textbf{Central Index/Bootstrap:} \\
				Server maintains peer/resource directory for faster discovery.
				
				\item \textbf{Direct Peer Communication:} \\
				Data exchanged directly once peers are located.
			\end{itemize}
		\end{column}
		
		\begin{column}{0.5\textwidth}
			\textbf{Key Features (cont.):}
			\begin{itemize}
				\item \textbf{Scalability:} \\
				Server reduces random peer search, allowing efficient growth.
				
				\item \textbf{Failure Tolerance:} \\
				Peers retain cached peer lists; network remains partially operational if server fails.
			\end{itemize}
			
			\vspace{10pt}
			\textbf{Summary:} \\
			Hybrid P2P improves efficiency and reliability by blending central management with peer autonomy.
		\end{column}
		
	\end{columns}
\end{frame}

\begin{frame}{Hybrid P2P: Pros and Cons}
	\vspace{20pt}
	\begin{columns}[t]
		
		\begin{column}{0.5\textwidth}
			\textbf{Advantages:}
			\begin{itemize}
				\item \textbf{Faster Resource Discovery:} \\
				Central server manages peer/resource index.
				
				\item \textbf{Reduced Network Load:} \\
				Avoids flooding, lowers bandwidth use.
				
				\item \textbf{Better Security \& Control:} \\
				Server enforces authentication, monitors activities.
				
				\item \textbf{Balanced Load:} \\
				Indexing reduces burden on individual peers.
			\end{itemize}
		\end{column}
		
		\begin{column}{0.5\textwidth}
			\textbf{Disadvantages:}
			\begin{itemize}
				\item \textbf{Single Point of Failure:} \\
				Server failure disrupts peer discovery.
				
				\item \textbf{Scalability Limits:} \\
				Central component can bottleneck large networks.
				
				\item \textbf{Privacy Concerns:} \\
				Server may log peer activities, reducing anonymity.
			\end{itemize}
			
			\vspace{10pt}
			\textbf{Summary:} \\
			Hybrid P2P offers efficiency and control, but at the cost of some decentralisation and privacy.
		\end{column}
		
	\end{columns}
\end{frame}

\begin{frame}{Examples of Hybrid P2P Networks}
	\vspace{20pt}
	\begin{columns}[t]
		
		\begin{column}{0.5\textwidth}
			\textbf{BitTorrent:}
			\begin{itemize}
				\item Uses central \textbf{tracker} to help peers find each other.
				\item File sharing happens directly between peers.
				\item Modern versions use \textbf{DHT} to reduce tracker dependency.
			\end{itemize}
		\end{column}
		
		\begin{column}{0.5\textwidth}
			\textbf{Skype (Early Versions):}
			\begin{itemize}
				\item Central directory server maps usernames to IP addresses.
				\item VoIP data transmitted directly peer-to-peer.
				\item Recent versions moved to more centralised cloud-based model.
			\end{itemize}
			
			\vspace{10pt}
			\textbf{Summary:} \\
			Hybrid P2P balances efficient peer discovery with direct communication, widely used in file sharing, VoIP, and content delivery.
		\end{column}
		
	\end{columns}
\end{frame}

\begin{frame}{Main Components in P2P Networks (1)}
	\vspace{20pt}
	\begin{columns}[t]
		
		\begin{column}{0.5\textwidth}
			\textbf{Peer:}  
			\begin{itemize}
				\item Acts as both client and server.
				\item Types:
				\begin{itemize}
					\item \textbf{Full Peer}: Stores complete data (e.g., blockchain full nodes).
					\item \textbf{Partial Peer}: Stores part of data (e.g., BitTorrent peers).
				\end{itemize}
				\item Shares, retrieves, and forwards resources.
			\end{itemize}
		\end{column}
		
		\begin{column}{0.5\textwidth}
			\textbf{Overlay Network:}  
			\begin{itemize}
				\item Virtual network over physical internet.
				\item Manages peer connections, routing, and resource distribution.
				\item Types:
				\begin{itemize}
					\item \textbf{Structured Overlay}: Uses DHT (e.g., Chord, Kademlia).
					\item \textbf{Unstructured Overlay}: Random connections (e.g., Gnutella, Freenet).
				\end{itemize}
			\end{itemize}
		\end{column}
		
	\end{columns}
\end{frame}



\begin{frame}{Main Components in P2P Networks (2)}
	\vspace{20pt}
	\begin{columns}[t]
		
		\begin{column}{0.5\textwidth}
			\textbf{Resource Discovery:}
			\begin{itemize}
				\item Enables peers to locate files/services.
				\item Methods:
				\begin{itemize}
					\item \textbf{Flooding}: Broadcasts search to all peers (e.g., Gnutella).
					\item \textbf{Random Walk}: Sends to random peers to reduce traffic.
					\item \textbf{DHT-Based}: Maps resources to peers (e.g., BitTorrent DHT).
				\end{itemize}
			\end{itemize}
		\end{column}
		
		\begin{column}{0.5\textwidth}
			\textbf{Message Routing:}
			\begin{itemize}
				\item Forwards queries/responses between peers.
				\item Methods:
				\begin{itemize}
					\item \textbf{Direct Connection}: Peer-to-peer communication.
					\item \textbf{Flooding}: Sends to all connected peers.
					\item \textbf{Random Walk}: Forwards randomly.
					\item \textbf{Routing Tables}: Used in structured P2P (e.g., Chord finger tables).
				\end{itemize}
			\end{itemize}
		\end{column}
		
	\end{columns}
\end{frame}


\begin{frame}{Main Components in P2P Networks (3)}
	\vspace{20pt}
	\begin{columns}[t]
		
		\begin{column}{0.5\textwidth}
			\textbf{Security \& Trust:}
			\begin{itemize}
				\item Mitigates risks: Sybil attacks, DoS, eavesdropping.
				\item Techniques:
				\begin{itemize}
					\item \textbf{Authentication \& Encryption}: SSL/TLS, end-to-end.
					\item \textbf{Reputation Trust}: Peer ratings (e.g., eMule scores).
					\item \textbf{Distributed Consensus}: PoW, PoS in blockchain.
				\end{itemize}
			\end{itemize}
		\end{column}
		
		\begin{column}{0.5\textwidth}
			\textbf{Summary:}
			\begin{itemize}
				\item Components:
				\begin{itemize}
					\item Peers
					\item Overlay Network
					\item Resource Discovery
					\item Message Routing
					\item Security Mechanisms
				\end{itemize}
				\item Enable decentralised, efficient, reliable systems.
				\item Applied in file sharing, communication, blockchain.
			\end{itemize}
		\end{column}
		
	\end{columns}
\end{frame}

\begin{frame}{Simple Chat Application}
	\vspace{20pt}
	The \textbf{peer-to-peer} (P2P) communication model allows each node in the network to act as both a client and a server. This architecture is popular because of its flexibility and scalability.
	
	\vspace{10pt}
	The application that will be developed in this module aims to provide a practical understanding of P2P through a simple chat. Here are the main features of the application:
	
	\begin{itemize}
		\item Uses the \textbf{P2P} model without a central server.
		\item Each \textbf{peer} can send and receive messages.
		\item Built using the \textbf{Rust} programming language.
		\item Utilizes the \textbf{Tokio} library for asynchronous communication.
	\end{itemize}
	
\end{frame}


\begin{frame}[fragile]{Peer-to-Peer Program: Part 1}
	\begin{lstlisting}[style=RustStyle]
		use tokio::net::{TcpListener, TcpStream};
		use tokio::io::{AsyncReadExt, AsyncWriteExt};
		use serde::{Serialize, Deserialize};
		use std::env;
		use std::error::Error;
		
		#[derive(Serialize, Deserialize)]
		struct Message {
			sender: String,
			content: String,
		}
		
		async fn start_server(address: &str) -> Result<(), Box<dyn Error>> {
			let listener = TcpListener::bind(address).await?;
			println!("Listening on {}", address);
		\end{lstlisting}
	\end{frame}
\begin{frame}[fragile]{Peer-to-Peer Program: Part 2}
	\vspace{20pt}
	\begin{lstlisting}[style=RustStyle]
		loop {
			let (mut socket, addr) = listener.accept().await?;
			println!("New connection from: {}", addr);
			
			tokio::spawn(async move {
				let mut buffer = vec![0; 1024];
				match socket.read(&mut buffer).await {
					Ok(size) if size > 0 => {
						let received_msg = String::from_utf8_lossy(&buffer[..size]);
						println!("Received: {}", received_msg);
					}
					_ => println!("Connection closed."),
				}
			});
		}
	}
	
	async fn send_message(peer_address: &str, sender_name: &str, content: &str) -> Result<(), Box<dyn Error>> {
		let mut stream = TcpStream::connect(peer_address).await?;
	\end{lstlisting}
\end{frame}
\begin{frame}[fragile]{Peer-to-Peer Program: Part 3}
	\begin{lstlisting}[style=RustStyle]
		let message = Message {
			sender: sender_name.to_string(),
			content: content.to_string(),
		};
		
		let serialized_msg = serde_json::to_string(&message)?;
		stream.write_all(serialized_msg.as_bytes()).await?;
		
		println!("Sent message: {}", serialized_msg);
		Ok(())
	}
\end{lstlisting}
\end{frame}

\begin{frame}[fragile]{Peer-to-Peer Program: Part 4a}
	\vspace{10pt}
	\begin{lstlisting}[style=RustStyle]
		#[tokio::main]
		async fn main() -> Result<(), Box<dyn std::error::Error>> {
			let args: Vec<String> = env::args().collect();
			
			if args.len() < 2 {
				println!("Usage:\n  Server mode: {} server <address>\n  Client mode: {} client <peer_address> <name> <message>", args[0], args[0]);
				return Ok(());
			}
			
			if args[1] == "server" {
				let address = args.get(2).cloned().unwrap_or_else(|| "127.0.0.1:8080".to_string());
				start_server(&address).await?;
			} else if args[1] == "client" {
			\end{lstlisting}
		\end{frame}
		
		\begin{frame}[fragile]{Peer-to-Peer Program: Part 4b}
			\vspace{10pt}
			\begin{lstlisting}[style=RustStyle]
				if args.len() < 5 {
					println!("Usage: {} client <peer_address> <name> <message>", args[0]);
					return Ok(());
				}
				let peer_address = &args[2];
				let sender_name = &args[3];
				let message = &args[4];
				send_message(peer_address, sender_name, message).await?;
			}
			
			Ok(())
		}
	\end{lstlisting}
\end{frame}

\begin{frame}[fragile]{Running the Server for Alice}
	\vspace{10pt}
	To start the server on Alice's side, use the following command:
	\begin{lstlisting}[language=bash]
		cargo run -- server 127.0.0.1:8080
	\end{lstlisting}
	
	\begin{itemize}
		\item \texttt{cargo run} will compile and run the Rust program.
		\item \texttt{server} specifies the server mode.
		\item \texttt{127.0.0.1:8080} is the IP address and port used by Alice.
	\end{itemize}
	
	\textbf{Output:}
	\begin{lstlisting}[language=bash]
		Listening on 127.0.0.1:8080
	\end{lstlisting}
\end{frame}

\begin{frame}[fragile]{Running the Server for Bob}
	\vspace{10pt}
	To start the server on Bob's side, run the following command:
	\begin{lstlisting}[language=bash]
		cargo run -- server 127.0.0.1:9090
	\end{lstlisting}
	
	\begin{itemize}
		\item Bob runs a server with the IP address \texttt{127.0.0.1} on the port \texttt{9090}.
		\item The command structure is the same as Alice's, just a different port.
	\end{itemize}
	
	\textbf{Output:}
	\begin{lstlisting}[language=bash]
		Listening on 127.0.0.1:9090
	\end{lstlisting}
	
\end{frame}

\begin{frame}[fragile]{Alice Sends a Message to Bob}
	\vspace{10pt}
	To send a message from Alice to Bob, use the command:
	\begin{lstlisting}[language=bash]
		cargo run -- client 127.0.0.1:9090 Alice "Hello Bob, how are you?"
	\end{lstlisting}
	
	\begin{itemize}
		\item \texttt{client} specifies the client mode.
		\item \texttt{127.0.0.1:9090} is Bob's server address.
		\item \texttt{Alice} is the sender's name.
		\item \texttt{"Hello Bob, how are you?"} is the message content.
	\end{itemize}
	
	\textbf{Output in Bob's Terminal:}
	\begin{lstlisting}[language=bash]
		New connection from: 127.0.0.1:xxxxx
		Received: {"sender":"Alice","content":"Hello Bob, how are you?"}
	\end{lstlisting}
\end{frame}

\begin{frame}[fragile]{Bob Sends a Message to Alice}
	\vspace{10pt}
	To send a message from Bob to Alice, use the command:
	\begin{lstlisting}[language=bash]
		cargo run -- client 127.0.0.1:8080 Bob "Hi Alice, I'm doing great! How about you?"
	\end{lstlisting}
	
	\begin{itemize}
		\item \texttt{client} specifies the client mode.
		\item \texttt{127.0.0.1:8080} is Alice's server address.
		\item \texttt{Bob} is the sender's name.
		\item \texttt{"Hi Alice, I'm doing great! How about you?"} is the body of the message.
	\end{itemize}
	
	\textbf{Output in Alice Terminal:}
	\begin{lstlisting}[language=bash]
		New connection from: 127.0.0.1:xxxxx
		Received: {"sender":"Bob","content":"Hi Alice, I'm doing great! How about you?"}
	\end{lstlisting}
\end{frame}

\section{Conclusion}

\begin{frame}{Conclusion}
	\vspace{20pt}
	\begin{itemize}
		\item \textbf{Peer-to-Peer (P2P)} architecture provides a decentralised approach where each node functions both as client and server.
		\item It enables direct communication, resource sharing, and improved fault tolerance without relying on central authority.
		\item Structured P2P networks offer efficient, scalable, and reliable resource discovery but at the cost of added complexity.
		\item Unstructured P2P networks are flexible and simple to maintain but face inefficiencies in search and scalability.
		\item Hybrid P2P models blend centralised and decentralised approaches to balance efficiency, control, and decentralisation.
		\item A simple P2P chat application built in \textbf{Rust} using \textbf{Tokio} and \textbf{Serde} demonstrates practical implementation of P2P communication.
	\end{itemize}
\end{frame}

\end{document}
