\documentclass[aspectratio=169, table]{beamer}


%\usepackage[beamertheme=./praditatheme]{Pradita}

\graphicspath{{../../images/}}

\usetheme{Pradita}

\subtitle{IF231303-Software Architecture  \vspace{10pt}}
\title{\huge Chapter-8:\\Serverless Architecture}
\author{\textbf{Yogi Valentino N, Steven Tanaka, Ryan Christensen}}

\begin{document}

\begin{frame}[plain]
	\maketitle
\end{frame}

	\begin{frame}
        \frametitle{Definisi Serverless Architecture}
%		\framesubtitle{\hspace{1cm}}
		\begin{itemize}
			\item \emph{Serverless architecture} merupakan pendekatan dalam desain \emph{software} yang mana developer tidak perlu pusing-pusing lagi mengelola infrastruktur seperti server.
			\item Developer bisa fokus dalam mengembangkan aplikasinya dan masalah server akan dikelola oleh penyedia layanan \emph{serverless} (provider).
			\item \emph{Serverless architecture} bukan berarti tanpa server sama sekali, tetapi memungkinkan untuk konfigurasi seminimal mungkin atau dikurangi.
		\end{itemize}
	\end{frame}

	\begin{frame}
    \frametitle{6 Key Components of Serverless Architecture}
	\framesubtitle{\hspace{1cm}}
		1. Function as a Service (FaaS)
		\begin{itemize}
			\item
			FaaS adalah blok bangunan dasar dari serverless, bertanggung jawab untuk menjalankan logika yang menentukan bagaimana sumber daya dialokasikan dalam suatu skenario tertentu.
			Tergantung pada lingkungan cloud yang digunakan, Anda dapat memilih layanan FaaS yang dirancang khusus seperti AWS Lambda untuk Amazon Web Services (AWS), Microsoft Azure Functions untuk Azure, Google Cloud Functions untuk Google Cloud Platform (GCP), dan IBM Cloud Functions untuk lingkungan pribadi atau hibrida.\&
			Fungsi-fungsi ini akan membaca database backend Anda ketika pengguna memicu suatu peristiwa dan mengekstrak serta mengirimkan respons.
		\end{itemize}
	\end{frame}

	\begin{frame}
        \frametitle{6 Key Components of Serverless Architecture}
			\framesubtitle{\hspace{1cm}}
		2. The Client Interface
		\begin{itemize}
			\item
			Antarmuka klien memainkan peran penting dalam fungsionalitas serverless. Anda tidak dapat memaksakan arsitektur serverless ke dalam aplikasi apa pun yang Anda pilih.
			Antarmuka harus mampu mendukung serangkaian permintaan yang singkat, interaksi tanpa keadaan, dan integrasi yang fleksibel.
			Antarmuka juga harus dirancang agar kompatibel dengan transfer data dalam volume yang sangat tinggi atau rendah.
		\end{itemize}
	\end{frame}

	\begin{frame}\frametitle{6 Key Components of Serverless Architecture}
			\framesubtitle{\hspace{1cm}}
		3. A web server on the Cloud
		\begin{itemize}
			\item Server web adalah tempat interaksi tanpa keadaan akan dimulai setelah pengguna memulainya dan sebelum layanan FaaS menghentikannya.
			Server web berbeda dari database backend, di mana informasi yang dikirimkan kepada pengguna disimpan. Sebagai contoh, misalkan Anda adalah penyedia konten video online.
			Dalam hal ini, server web adalah tempat permintaan pengguna, skrip, respons FaaS, dll., dihosting sebelum dihentikan sesuai dengan sifat sementara dari serverless.
			Di sisi lain, konten video akan disimpan di penyimpanan backend, menunggu untuk diambil sesuai permintaan pengguna.
		\end{itemize}
	\end{frame}

	\begin{frame}\frametitle{6 Key Components of Serverless Architecture}
			\framesubtitle{\hspace{1cm}}
		4. A Security Service

		Keamanan adalah elemen kunci dari operasi serverless karena:
		\begin{itemize}
			\item a. Aplikasi menangani ribuan permintaan secara bersamaan. Setiap permintaan harus diotentikasi sebelum mengirimkan respons.
			\item b. Karena sifatnya yang tidak memiliki keadaan, riwayat interaksi sebelumnya tidak disimpan. Aplikasi tidak dapat mengandalkan interaksi sebelumnya untuk memvalidasi interaksi di masa depan.


		\end{itemize}
	\end{frame}

		\begin{frame}\frametitle{6 Key Components of Serverless Architecture}
			\framesubtitle{\hspace{1cm}}
		4. A Security Service

		Keamanan adalah elemen kunci dari operasi serverless karena:
		\begin{itemize}
			\item c. Model serverless membuat transparansi dan pemantauan menjadi lebih sulit. Anda harus mendapatkan informasi keamanan dari jutaan peristiwa yang dicatat setiap hari.
			\item d. Sifat terdistribusi dari arsitektur serverless berarti ada beberapa layanan dan vendor yang terlibat. Seluruh lanskap harus diamankan.

		\end{itemize}
	\end{frame}


	\begin{frame}\frametitle{6 Key Components of Serverless Architecture}
			\framesubtitle{\hspace{1cm}}
		4. A Security Service

		Keamanan adalah elemen kunci dari operasi serverless karena:
		\begin{itemize}


			\item e. Biasanya, aplikasi serverless menggunakan layanan token, di mana kredensial sementara dihasilkan untuk pengguna dan dapat digunakan untuk memanggil fungsi. Anda juga dapat mengintegrasikan layanan manajemen identitas dan akses yang siap digunakan dengan serverless ke dalam aplikasi Anda. Sebagai contoh, AWS Cognito bekerja dengan AWS Lambda untuk mengautentikasi identitas pengguna melalui SSO atau jaringan sosial. Anda mungkin menemukan layanan serupa untuk vendor cloud pilihan Anda.
		\end{itemize}
	\end{frame}

	\begin{frame}\frametitle{6 Key Components of Serverless Architecture}
			\framesubtitle{\hspace{1cm}}
		5. Database Backend
		\begin{itemize}
			\item Database backend adalah tempat data disimpan. Ini adalah komponen yang paling penting dari arsitektur serverless karena menyimpan informasi yang dikirimkan kepada pengguna.
			Database backend juga bertanggung jawab untuk mengelola data yang dikirimkan oleh pengguna.
			Ini adalah komponen yang paling penting dari arsitektur serverless karena menyimpan informasi yang dikirimkan kepada pengguna.
			Database backend juga bertanggung jawab untuk mengelola data yang dikirimkan oleh pengguna.
		\end{itemize}
	\end{frame}

	\begin{frame}\frametitle{6 Key Components of Serverless Architecture}
			\framesubtitle{\hspace{1cm}}
		6. API Gateway
		\begin{itemize}
			\item API Gateway bertindak sebagai gerbang untuk menghubungkan pengguna dengan fungsi yang sesuai.
			Ini adalah komponen yang sangat penting dari arsitektur serverless karena memungkinkan Anda untuk mengontrol lalu lintas dan memastikan bahwa permintaan yang masuk diteruskan ke fungsi yang sesuai.
			API Gateway juga memungkinkan Anda untuk mengatur kebijakan akses dan otentikasi untuk memastikan bahwa pengguna yang sah dapat mengakses fungsi yang sesuai.
		\end{itemize}
	\end{frame}

	\begin{frame}\frametitle{6 Key Components of Serverless Architecture}
		\framesubtitle{\hspace{1cm}}
		\begin{itemize}
			\item Dalam keseluruhan arsitektur serverless, FaaS berperan sebagai eksekutor logika yang menentukan alokasi sumber daya, antarmuka klien berfungsi sebagai antarmuka pengguna, server web di cloud menangani interaksi tanpa keadaan, layanan keamanan menjaga keamanan dan otentikasi, database backend menyimpan data yang akan dikirimkan kepada pengguna, dan API gateway menghubungkan semua komponen tersebut.
			Dengan memahami komponen-komponen ini, Anda dapat merancang dan mengimplementasikan aplikasi serverless yang efisien dan skalabel di lingkungan cloud pilihan Anda.
		\end{itemize}
	\end{frame}
	\begin{frame}
		\begin{figure}
			\includegraphics[width=0.5\linewidth]{komponen.jpg}
			\centering
			\caption{6 Key Components of Serverless Architecture}
		\end{figure}
	\end{frame}

	\begin{frame}\frametitle{Kegunaan Serverless Architecture}
%		\framesubtitle{\hspace{1cm}}
		Serverless architecture adalah sebuah model komputasi di awan yang mana penyedia layanan awan bertanggung jawab dalam mengelola infrastruktur dan memperuntukan sumber daya komputasi yang dibutuhkan secara otomatis, tanpa pengguna perlu mengurus atau merawat server. Terdapat beberapa keuntungan dari penggunaan arsitektur serverless, diantaranya adalah:

	\end{frame}

	\begin{frame}\frametitle{Kegunaan Serverless Architecture (2)}
		\framesubtitle{\hspace{1cm}}

		Penilaian untuk setiap karakteristik berdasarkan kecenderungan alami untuk implementasi tipikal pola layered.

		\begin{itemize}
			\item optimisasi biaya dengan hanya membayar untuk sumber daya yang digunakan
			\item memungkinkan pengembangan aplikasi yang lebih cepat
			\item skalabilitas yang mudah untuk mengakomodasi perubahan permintaan
			\item serta perawatan dan pemeliharaan yang lebih mudah karena dikelola oleh penyedia layanan awan.
		\end{itemize}
	\end{frame}

	\begin{frame}\frametitle{Kelebihan Serverless Architecture}
%		\framesubtitle{\hspace{1cm}}
		Kelebihan dari Serverless Architecture:
		\begin{itemize}
			\item \textbf{Skalabilitas:} Dalam arsitektur serverless, aplikasi dapat dengan mudah ditingkatkan kapasitasnya untuk menangani permintaan yang berfluktuasi, sehingga dapat mengurangi pengeluaran untuk infrastruktur yang tidak terpakai.
			\item \textbf{Biaya operasional yang rendah:} Dalam serverless, pengguna hanya membayar untuk sumber daya yang mereka gunakan, yang dapat mengurangi biaya operasional secara signifikan.

		\end{itemize}
	\end{frame}

	\begin{frame}\frametitle{Kelebihan Serverless Architecture (2)}
		\framesubtitle{\hspace{1cm}}
		Kelebihan dari Serverless Architecture:
		\begin{itemize}

			\item \textbf{Fokus pada pengembangan aplikasi:} Dalam serverless, pengembang tidak perlu khawatir mengurus infrastruktur server dan dapat fokus pada pengembangan aplikasi.
			\item \textbf{Perawatan dan pemeliharaan yang mudah:} Dalam serverless, penyedia layanan awan bertanggung jawab atas perawatan dan pemeliharaan infrastruktur, sehingga pengguna tidak perlu memikirkan pembaruan sistem operasi atau patch keamanan.
		\end{itemize}
	\end{frame}

	\begin{frame}\frametitle{Kekurangan Serverless Architecture}
		\framesubtitle{\hspace{1cm}}
		Kekurangan dari Serverless Architecture:
		\begin{itemize}
			\item \textbf{Keterbatasan dalam penggunaan:} Serverless mungkin tidak cocok untuk semua jenis aplikasi, terutama jika aplikasi memerlukan kontrol tinggi atas infrastruktur dan lingkungan di mana aplikasi berjalan.
			\item \textbf{Ketergantungan pada penyedia layanan awan:} Serverless membuat pengguna sangat bergantung pada penyedia layanan awan, sehingga jika terjadi masalah atau gangguan pada layanan, aplikasi dapat mengalami downtime yang signifikan.

		\end{itemize}
	\end{frame}

	\begin{frame}\frametitle{Kekurangan Serverless Architecture (2)}
			\framesubtitle{\hspace{1cm}}
		Kekurangan dari Serverless Architecture:
		\begin{itemize}

			\item \textbf{Pengaturan konfigurasi yang kompleks:} Serverless dapat memiliki konfigurasi yang kompleks dan memerlukan pengaturan yang cermat untuk memastikan aplikasi berjalan dengan baik.
			\item \textbf{Performa yang tidak stabil:} Serverless dapat mengalami performa yang tidak stabil jika pengguna tidak melakukan penyesuaian yang cermat dalam skala dan konfigurasi aplikasi.
		\end{itemize}
	\end{frame}

	\begin{frame}\frametitle{Penerapan Serverless Architecture}
%		\framesubtitle{\hspace{1cm}}
		\begin{itemize}
			\item \textbf{Web applications:} Pengembang dapat membangun dan menerapkan aplikasi web menggunakan arsitektur tanpa server, tanpa harus mengelola server atau infrastruktur. Mereka dapat menggunakan layanan seperti AWS Lambda, Google Cloud Functions, atau Azure Functions untuk menulis kode yang merespons kejadian, seperti permintaan HTTP, dan berjalan sesuai permintaan.
			\item \textbf{Data processing:} Serverless Architecture dapat digunakan untuk tugas pemrosesan data seperti transformasi data, pembersihan, dan analisis. Pengembang dapat menggunakan layanan seperti AWS Glue, Google Cloud Dataflow, atau Azure Stream Analytics untuk memproses data tanpa server, tanpa harus mengelola server atau infrastruktur.

		\end{itemize}
	\end{frame}


	\begin{frame}\frametitle{Penerapan Serverless Architecture (2)}
		\framesubtitle{\hspace{1cm}}
		\begin{itemize}
			\item \textbf{Chatbots:} Pengembang dapat membangun chatbot menggunakan serverless, dengan menulis kode yang merespons acara obrolan, seperti pesan pengguna. Mereka dapat menggunakan layanan seperti AWS Lex, Google Cloud Dialogflow, atau Azure Bot Service untuk membuat dan menerapkan chatbot yang berjalan sesuai permintaan.
			\item \textbf {IoT applications:} Serverless Architecture dapat digunakan untuk aplikasi IoT, dengan memungkinkan pengembang menulis kode yang merespons kejadian dari perangkat IoT, seperti pembacaan sensor. Mereka dapat menggunakan layanan seperti AWS IoT, Google Cloud IoT Core, atau Azure IoT Hub untuk membangun dan menerapkan aplikasi IoT tanpa server.
		\end{itemize}
	\end{frame}
\end{document}