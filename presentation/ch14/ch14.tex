\documentclass[aspectratio=169, table]{beamer}

\usepackage{listings}
\usepackage{tikz}
\usetikzlibrary{ fit, shapes.geometric, arrows.meta, positioning}
\usepackage{array}
\usepackage{float}
\usepackage{colortbl} 

\lstdefinestyle{RustStyle}{
	language=Java,
	morekeywords={println, Ok, async, fn, main, use, let, mut},
	basicstyle=\ttfamily\scriptsize,
	keywordstyle=\color{blue},
	commentstyle=\color{gray},
	stringstyle=\color{red},
	breaklines=true,
	showstringspaces=false,
	tabsize=2,
	captionpos=b,
	numbers=left,
	numberstyle=\tiny\color{gray},
	frame=lines,
	backgroundcolor=\color{lightgray!10},
	comment=[l]{//},
	morecomment=[s]{/*}{*/},
	commentstyle=\color{gray}\ttfamily,
	string=[s]{'}{'},
	morestring=[s]{"}{"},
	stringstyle=\color{teal}\ttfamily,
	%	showstringspaces=false
	literate=
	{\{}{{\textcolor{red}{\{}}}1
	{\}}{{\textcolor{red}{\}}}}1
	{:}{{\textcolor{red}{:}}}1
	{=}{{\textcolor{red}{=}}}1
	{.}{{\textcolor{red}{.}}}1
	{]}{{\textcolor{red}{]}}}1
	{[}{{\textcolor{red}{[}}}1
	{\#}{{\textcolor{red}{\#}}}1
	{;}{{\textcolor{red}{;}}}1
	{?}{{\textcolor{red}{?}}}1
	{!}{{\textcolor{red}{!}}}1
}

%\usepackage[beamertheme=./praditatheme]{Pradita}

\usetheme{Pradita}

\lstdefinestyle{RustStyle}{
	language=Java,
	morekeywords={println, Ok, async, fn, main, use, let, mut},
	basicstyle=\ttfamily\footnotesize,
	keywordstyle=\color{blue},
	commentstyle=\color{gray},
	stringstyle=\color{red},
	breaklines=true,
	showstringspaces=false,
	tabsize=2,
	captionpos=b,
	numbers=left,
	numberstyle=\tiny\color{gray},
	frame=lines,
	backgroundcolor=\color{lightgray!10},
	comment=[l]{//},
	morecomment=[s]{/*}{*/},
	commentstyle=\color{gray}\ttfamily,
	string=[s]{'}{'},
	morestring=[s]{"}{"},
	stringstyle=\color{teal}\ttfamily,
	%	showstringspaces=false
	literate=
	{\{}{{\textcolor{red}{\{}}}1
	{\}}{{\textcolor{red}{\}}}}1
	{:}{{\textcolor{red}{:}}}1
	{=}{{\textcolor{red}{=}}}1
	{.}{{\textcolor{red}{.}}}1
	{]}{{\textcolor{red}{]}}}1
	{[}{{\textcolor{red}{[}}}1
	{\#}{{\textcolor{red}{\#}}}1
	{;}{{\textcolor{red}{;}}}1
	{?}{{\textcolor{red}{?}}}1
	{!}{{\textcolor{red}{!}}}1
}

\lstdefinelanguage{bash} {
	keywords={},
	basicstyle=\ttfamily\small,
	keywordstyle=\color{blue}\bfseries,
	ndkeywords={iex},
	ndkeywordstyle=\color{purple}\bfseries,
	sensitive=true,
	commentstyle=\color{gray},
	stringstyle=\color{red},
	numbers=left,
	numberstyle=\tiny\color{gray},
	breaklines=true,
	frame=lines,
	backgroundcolor=\color{lightgray!10},
	tabsize=2,
	comment=[l]{\#},
	morecomment=[s]{/*}{*/},
	commentstyle=\color{gray}\ttfamily,
	stringstyle=\color{purple}\ttfamily,
	showstringspaces=false
}

\lstdefinelanguage{docker} {
	keywords={FROM, EXPOSE, RUN, ARG, ENTRYPOINT, EXPOSE, WORKDIR, COPY, as},
	basicstyle=\ttfamily\small,
	keywordstyle=\color{blue}\bfseries,
	ndkeywords={iex},
	ndkeywordstyle=\color{purple}\bfseries,
	sensitive=true,
	commentstyle=\color{gray},
	stringstyle=\color{red},
	numbers=left,
	numberstyle=\tiny\color{gray},
	breaklines=true,
	frame=lines,
	backgroundcolor=\color{lightgray!10},
	tabsize=2,
	comment=[l]{\#},
	%	morecomment=[s]{}{},
	commentstyle=\color{gray}\ttfamily,
	stringstyle=\color{purple}\ttfamily,
	showstringspaces=false
}

\lstdefinelanguage{yaml} {
	keywords={ },
	basicstyle=\ttfamily\small,
	keywordstyle=\color{blue}\bfseries,
	ndkeywords={iex},
	ndkeywordstyle=\color{purple}\bfseries,
	sensitive=true,
	commentstyle=\color{gray},
	stringstyle=\color{red},
	numbers=left,
	numberstyle=\tiny\color{gray},
	breaklines=true,
	frame=lines,
	backgroundcolor=\color{lightgray!10},
	tabsize=2,
	comment=[l]{\#},
	%	morecomment=[s]{}{},
	commentstyle=\color{gray}\ttfamily,
	stringstyle=\color{purple}\ttfamily,
	showstringspaces=false
}


\title{\Huge DevOps\\\vspace{30pt}}
\subtitle{IF231303-Software Architecture}
\author{\textbf{Alfa Yohannis}}
\begin{document}
	
	\frame{\titlepage}
	
	
	\begin{frame}[fragile]
		\frametitle{Contents}
		\vspace{20pt}
		\begin{columns}[t]
			\column{0.5\textwidth}
			\tableofcontents[sections={1-5}]
			
			\column{0.5\textwidth}
			\tableofcontents[sections={6-10}]
		\end{columns}
	\end{frame}
	
\section{Pendahuluan}

\begin{frame}[fragile]{Pendahuluan: DevOps dan GitHub Actions}
	\vspace{20pt}
	DevOps menyatukan proses \textbf{pengembangan} dan \textbf{operasional} ke dalam satu siklus berkelanjutan—mulai dari perencanaan hingga pemantauan—untuk meningkatkan kecepatan, kualitas, dan keandalan pengiriman perangkat lunak.
	
	\vspace{6pt}
	\textbf{Fokus Utama:}
	\begin{itemize}
		\item Automasi proses melalui CI/CD (Continuous Integration/Deployment)
		\item Kolaborasi dan tanggung jawab bersama antar tim
		\item Adaptasi cepat terhadap perubahan dan kebutuhan bisnis
	\end{itemize}
	
	\vspace{6pt}
	Dalam praktiknya, DevOps didukung oleh platform seperti \textbf{GitHub Actions} yang mengintegrasikan workflow otomatis langsung ke repositori. Pendekatan ini membantu tim membangun sistem yang \textit{reliable}, terotomatisasi, dan siap produksi.
\end{frame}

\section{Prinsip dan Nilai Utama DevOps}

\begin{frame}[fragile]{Prinsip dan Nilai Utama DevOps}
	\vspace{20pt}
	
	DevOps adalah filosofi kerja yang menekankan \textbf{kolaborasi lintas fungsi}, \textbf{automasi}, dan \textbf{perbaikan berkelanjutan}. Tujuannya adalah menghilangkan hambatan antara tim Dev dan Ops, mempercepat penyampaian fitur, dan menjaga kestabilan sistem dalam perubahan yang konstan.
	
	\vspace{6pt}
	\textbf{Empat Pilar DevOps:}
	\begin{itemize}
		\item \textbf{Kolaborasi Dev \& Ops:} tim lintas fungsi, tanggung jawab bersama, komunikasi terbuka.
		\item \textbf{Automasi dan CI/CD:} integrasi dan deployment otomatis untuk mempercepat rilis dan mengurangi error.
		\item \textbf{Monitoring dan Feedback:} observability (logs, metrics, tracing) serta umpan balik pengguna untuk iterasi.
		\item \textbf{Budaya Eksperimen:} uji coba aman, perbaikan berkelanjutan, retrospektif, dan pembelajaran dari insiden.
	\end{itemize}
\end{frame}



\section{Siklus Hidup DevOps}

\begin{frame}[fragile]{Siklus Hidup DevOps}
	\vspace{20pt}
	
	Siklus hidup DevOps mulai dari perencanaan hingga pemantauan. Tidak bersifat linear, melainkan iteratif dan otomatis, memungkinkan perubahan kode, pengujian, dan rilis berlangsung cepat dan stabil, sekaligus responsif terhadap perubahan.
	
	\vspace{6pt}
	\textbf{Tahapan Utama:}
	\begin{itemize}
		\item \textbf{Plan:} susun backlog dan rencana rilis (\textit{GitHub Issues}, Jira).
		\item \textbf{Develop:} kolaborasi melalui Git, integrasi rutin (CI).
		\item \textbf{Build:} kompilasi dan kemas kode ke artefak deploy.
		\item \textbf{Test:} jalankan pengujian otomatis dari unit hingga E2E.
		\item \textbf{Release:} rilis otomatis bertahap (blue-green, canary).
		\item \textbf{Deploy:} penyebaran ke staging atau production.
		\item \textbf{Operate:} kelola performa, konfigurasi, dan stabilitas.
		\item \textbf{Monitor:} pantau log, metrics, dan tracing untuk feedback.
	\end{itemize}
\end{frame}

\section{Penerapan Continuous Integration dan Deployment}

\begin{frame}[fragile]{Penerapan CI/CD dengan GitHub Actions}
	\vspace{20pt}
	
	\begin{columns}[T]
		\begin{column}{0.56\textwidth}
			\textbf{CI/CD dan Otomasi DevOps}
			
			CI/CD adalah praktik DevOps yang menguji dan menyebarkan kode otomatis. CI memastikan perubahan diuji sejak awal. CD mempercepat rilis tanpa intervensi manual. CI/CD mendukung rilis cepat, stabil, dan efisiensi tinggi.
			
			\vspace{6pt}
			\textbf{GitHub Actions}
			\begin{itemize}
				\item Workflow dalam file YAML di \texttt{.github/workflows/}.
				\item Dipicu oleh \texttt{push}, \texttt{pull\_request}, atau \texttt{schedule}.
				\item Atur langkah build, test, dan deploy secara deklaratif.
			\end{itemize}
		\end{column}
		
		\begin{column}{0.44\textwidth}
			\textbf{Strategi CI/CD Efektif}
			\begin{itemize}
				\item Gunakan branching strategy yang rapi (Git Flow, trunk-based).
				\item Automasi end-to-end dari commit hingga deploy.
				\item Pisahkan environment: dev, staging, production.
				\item Gunakan semantic versioning dan approval gate.
			\end{itemize}
			
			\vspace{8pt}
			Strategi ini menjadikan pipeline CI/CD lebih modular, andal, dan siap untuk skala besar.
		\end{column}
	\end{columns}
\end{frame}



\section{Infrastructure as Code dengan GitHub Actions}

\begin{frame}[fragile]{Infrastructure as Code (IaC) \& GitHub Actions}
	\vspace{20pt}
	
	\begin{columns}[T]
		\begin{column}{0.6\textwidth}
			\textbf{Konsep IaC}
			
			Infrastructure as Code (IaC) mengelola infrastruktur dengan file konfigurasi yang dapat diperlakukan seperti kode. Proses provisioning jadi otomatis, konsisten, dan mudah direproduksi.
			
			\vspace{6pt}
			\textbf{Peran GitHub Actions}
			\begin{itemize}
				\item Menjalankan tool seperti Terraform, Ansible, Pulumi.
				\item Workflow dipicu saat file konfigurasi berubah.
				\item Automasi init, plan, apply dalam satu pipeline.
			\end{itemize}
		\end{column}
		
		\begin{column}{0.4\textwidth}
			\textbf{Integrasi Cloud dan Container}
			\begin{itemize}
				\item Build dan push image Docker ke registry (Docker Hub, ECR).
				\item Deploy otomatis ke Kubernetes (EKS, GKE, AKS).
				\item Gunakan secrets untuk autentikasi aman.
			\end{itemize}
			
			\vspace{8pt}
			Dengan GitHub Actions, infrastruktur dan aplikasi dapat diproses otomatis end-to-end tanpa intervensi manual.
		\end{column}
	\end{columns}
\end{frame}

%\section{Monitoring dan Observability di GitHub Actions}
%
%\begin{frame}[fragile]{Monitoring dan Observability di GitHub Actions}
%	\vspace{20pt}
%	
%	\begin{columns}[T]
%		\begin{column}{0.6\textwidth}
%			\textbf{Status dan Log}
%			\begin{itemize}
%				\item Log eksekusi otomatis tiap job dan step.
%				\item Status visual: ✅ sukses, ❌ gagal, ⏳ berjalan.
%				\item Gunakan anotasi log untuk debugging cepat.
%				\item Log dapat diakses dan diunduh dari tab \texttt{Actions}.
%			\end{itemize}
%			
%			\vspace{4pt}
%			\textbf{Notifikasi Otomatis}
%			\begin{itemize}
%				\item Kirim status ke Slack, email, atau Discord.
%				\item Gunakan webhook atau GitHub Actions khusus.
%			\end{itemize}
%		\end{column}
%		
%		\begin{column}{0.4\textwidth}
%			\textbf{Integrasi Eksternal}
%			\begin{itemize}
%				\item Integrasi dengan Datadog, Grafana, New Relic.
%				\item Hubungkan Sentry/Rollbar untuk pelacakan error.
%				\item Ekspos metrik lokal via Prometheus (self-hosted).
%			\end{itemize}
%			
%			\vspace{6pt}
%			Observability menjadikan GitHub Actions bukan hanya otomasi, tapi alat pemantauan aktif dan umpan balik berkelanjutan.
%		\end{column}
%	\end{columns}
%\end{frame}
%
%
%\section{Tantangan dan Praktik Terbaik GitHub Actions}
%
%\begin{frame}[fragile]{Tantangan dan Best Practices GitHub Actions}
%	\vspace{20pt}
%	
%	\begin{columns}[T]
%		\begin{column}{0.66\textwidth}
%			\textbf{Keamanan dan Secrets}
%			\begin{itemize}
%				\item Simpan secrets dengan \texttt{secrets.\{NAMA\}} secara terenkripsi.
%				\item Hindari mencetak secrets di log (\texttt{::add-mask::}).
%				\item Batasi akses dan gunakan Environment Protection Rules.
%				\item Rutin rotasi secrets untuk mencegah kebocoran.
%			\end{itemize}
%			
%			\vspace{4pt}
%			\textbf{Optimasi Pipeline}
%			\begin{itemize}
%				\item Gunakan cache untuk dependensi.
%				\item Jalankan workflow hanya saat file tertentu berubah.
%				\item Pisahkan job agar berjalan paralel.
%				\item Gunakan reusable workflow untuk hindari duplikasi.
%				\item Hindari self-hosted runner jika tak diperlukan.
%			\end{itemize}
%		\end{column}
%		
%		\begin{column}{0.34\textwidth}
%			\textbf{Composite Actions}
%			\begin{itemize}
%				\item Bungkus langkah berulang jadi satu action.
%				\item Definisikan dengan \texttt{action.yml} dalam repo.
%				\item Permudah reuse lintas repositori dan tim.
%			\end{itemize}
%			
%			\vspace{6pt}
%			Praktik terbaik ini membuat pipeline lebih aman, efisien, dan mudah dikelola, sekaligus mendukung skala besar dan kolaborasi tim.
%		\end{column}
%	\end{columns}
%\end{frame}

\section{Studi Kasus Penggunaan GitHub Actions}

\begin{frame}[fragile]{Studi Kasus: CI/CD Aplikasi Rust Sederhana}
	\vspace{20pt}
	Contoh Kode: \url{https://github.com/alfa-yohannis/rust-git-workflow-example}
	\begin{columns}[T]
		\begin{column}{0.5\textwidth}
			\textbf{Deskripsi Aplikasi}
			\begin{itemize}
				\item Web service berbasis \texttt{actix-web}.
				\item Endpoint \texttt{/sum} menerima \texttt{a} dan \texttt{b}.
				\item Output: hasil penjumlahan dalam teks.
			\end{itemize}
			
			\vspace{4pt}
			\textbf{Otomasi Utama}
			\begin{itemize}
				\item Build otomatis via \texttt{cargo build}.
				\item Uji fungsi dengan \texttt{cargo test}.
				\item Deploy jika lulus semua tahapan.
			\end{itemize}
		\end{column}
		
		\begin{column}{0.5\textwidth}
			\textbf{Pipeline GitHub Actions}
			\begin{itemize}
				\item Dipicu oleh \texttt{push} ke \texttt{main}.
				\item Gunakan \texttt{actions-rs/toolchain}, \texttt{actions/cache}.
				\item Caching \texttt{.cargo} dan \texttt{target} untuk efisiensi.
				\item Workflow berjalan otomatis di \texttt{ubuntu-latest}.
			\end{itemize}
			
			
		\end{column}
	\end{columns}
\vspace{6pt}
Studi kasus ini menunjukkan penerapan pipeline CI/CD sederhana menggunakan GitHub Actions untuk proyek berbasis Rust.
\end{frame}

\begin{frame}[fragile]{Struktur Proyek Rust}
\vspace{20pt}
\begin{lstlisting}[language=bash]
.
|-- Cargo.lock
|-- Cargo.toml
|-- Dockerfile
|-- src
|   |-- handlers.rs
|   |-- lib.rs
|   |-- main.rs
|-- target
|   |-- debug
|   |-- tmp
|-- tests
|   |-- sum_tests.rs
\end{lstlisting}

Kode sumber ada di \texttt{src/}, konfigurasi di \texttt{Cargo.toml}, dan pengujian terpisah di \texttt{tests/}.
\end{frame}


\begin{frame}[fragile]{Konfigurasi Cargo.toml}
\begin{lstlisting}[language=bash]
[package]
name = "rust-example"
version = "0.1.0"
edition = "2021"

[dependencies]
actix-rt = "2.10.0"
actix-web = "4.11.0"
\end{lstlisting}

\texttt{Cargo.toml} berisi metadata proyek dan dependensi seperti \texttt{actix-web} dan \texttt{actix-rt} yang diperlukan untuk membangun aplikasi web asynchronous di Rust.
\end{frame}


\begin{frame}[fragile]{Handler untuk Penjumlahan}
\begin{lstlisting}[style=RustStyle]
use actix_web::{web, HttpResponse, Responder};

pub async fn sum(query: web::Query<std::collections::HashMap<String, String>>) -> impl Responder {
	let a= query.get("a").and_then(|v| v.parse::<i32>().ok()).unwrap_or(0);
	let b= query.get("b").and_then(|v| v.parse::<i32>().ok()).unwrap_or(0);
	HttpResponse::Ok().body(format!("{}", a + b))
}
\end{lstlisting}

Fungsi \texttt{sum} membaca parameter \texttt{a} dan \texttt{b} dari URL query string, menghitung jumlahnya, dan mengembalikannya sebagai respons HTTP 200.
\end{frame}

\begin{frame}[fragile]{Modul Utama Library}
\begin{lstlisting}[style=RustStyle, caption={src/lib.rs}]
pub mod handlers;
\end{lstlisting}

File \texttt{lib.rs} mendeklarasikan modul \texttt{handlers} agar dapat digunakan oleh file lain seperti \texttt{main.rs} atau modul pengujian.
\end{frame}

\begin{frame}[fragile]{Titik Masuk Aplikasi}
	\vspace{20pt}
\begin{lstlisting}[style=RustStyle]
use actix_web::{web, App, HttpServer};

mod handlers;

#[actix_web::main]
async fn main() -> std::io::Result<()> {
	HttpServer::new(|| {
		App::new().route("/sum", web::get().to(handlers::sum))
	})
	.bind("0.0.0.0:8080")?
	.run()
	.await
}
\end{lstlisting}

\texttt{main.rs} menjalankan server Actix Web dan mengaitkan rute \texttt{/sum} ke handler \texttt{sum} pada port \texttt{8080}.
\end{frame}

\begin{frame}[fragile]{Pengujian Handler \texttt{sum}}
	\vspace{20pt}
\begin{lstlisting}[style=RustStyle]
#[actix_rt::test]
async fn test_sum_valid_params() {
	let app = test::init_service(App::new().route("/sum", web::get().to(handlers::sum))).await;
	let req = test::TestRequest::get().uri("/sum?a=10&b=15").to_request();
	let resp = test::call_service(&app, req).await;
	assert_eq!(resp.status(), StatusCode::OK);
	let body = to_bytes(resp.into_body()).await.unwrap();
	assert_eq!(body, "25");
}
...
async fn test_sum_missing_params() { ... }
\end{lstlisting}

Fungsi \texttt{test\_sum\_valid\_params} menguji hasil penjumlahan parameter \texttt{a} dan \texttt{b}. Fungsi lain menguji kondisi dengan parameter parsial, tambahan, atau kosong, untuk memastikan handler tetap merespons dengan benar.
\end{frame}



\begin{frame}[fragile]{Automasi Testing dan Linting (1)}
\vspace{10pt}
\begin{columns}[T]
\begin{column}{0.5\textwidth}
\begin{lstlisting}[language=yaml, firstnumber=1]
name: Rust CI

on:
	push:
	branches: [ main ]
	pull_request:
	branches: [ main ]

jobs:
	build-and-test:
	runs-on: ubuntu-latest
\end{lstlisting}
\end{column}

\begin{column}{0.5\textwidth}
\begin{lstlisting}[language=yaml, firstnumber=12]
steps:
	- name: Checkout source code
		uses: actions/checkout@v4

	- name: Set up Rust
		uses: actions-rs/toolchain@v1
		with:
			toolchain: stable
			override: true
\end{lstlisting}
\end{column}
\end{columns}

\vspace{6pt}
Workflow ini dijalankan saat ada perubahan di branch \texttt{main} dan menyiapkan lingkungan Rust.
\end{frame}

\begin{frame}[fragile]{Automasi Testing dan Linting (2)}
\vspace{10pt}
\begin{columns}[T]
\begin{column}{0.5\textwidth}
\begin{lstlisting}[language=yaml, firstnumber=21]
	- name: Install system dependencies
		run: sudo apt-get update &&
		sudo apt-get install -y
		pkg-config libssl-dev
	
	- name: Build project
		run: cargo build --verbose
		working-directory: rust-example
\end{lstlisting}
\end{column}

\begin{column}{0.5\textwidth}
\begin{lstlisting}[language=yaml, firstnumber=29]
	- name: Run tests
		run: cargo test --verbose
		working-directory: rust-example
\end{lstlisting}
\end{column}
\end{columns}

\vspace{6pt}
Bagian ini melakukan instalasi dependensi, membangun proyek, dan menjalankan pengujian otomatis.
\end{frame}

\begin{frame}[fragile]{Kesimpulan}
	\vspace{20pt}
	
	\begin{columns}[T]
		\begin{column}{0.5\textwidth}
			\begin{itemize}
				\item GitHub Actions otomatisasi CI/CD end-to-end
				\item Mendukung build, lint, test, dan deploy
				\item Cocok untuk proyek kecil maupun besar
				\item Workflow mudah dikustomisasi
			\end{itemize}
		\end{column}
		
		\begin{column}{0.5\textwidth}
			\begin{itemize}
%				\item Dukungan secrets dan approval rules
%				\item Terintegrasi dengan banyak platform
				\item Mendukung berbagai bahasa pemrograman
				\item Praktik DevOps jadi lebih terstruktur
			\end{itemize}
		\end{column}
	\end{columns}
	
\end{frame}



\end{document}
