\documentclass{beamer}
\usepackage{graphicx}

\title{Microservices}
\author{Alfred Gerald Thendiwijaya, Lucky Rusandana, Inzaghi Posuma Al Kahfi}

\begin{document}
	
	\frame{\titlepage}
	
	\begin{frame}
		\frametitle{Definisi Microservices}
		Microservices adalah sebuah arsitektur perangkat lunak yang membagi sebuah aplikasi besar menjadi beberapa komponen kecil yang independen dan dapat berkomunikasi dengan satu sama lain melalui antarmuka yang didefinisikan secara jelas. Setiap komponen atau layanan (service) dalam arsitektur microservices memiliki tugas dan tanggung jawab tertentu yang dapat dijalankan secara mandiri dan dapat diubah tanpa mempengaruhi layanan lain dalam aplikasi. Dalam arsitektur microservices, komunikasi antara layanan biasanya dilakukan melalui protokol HTTP atau pesan. Kelebihan arsitektur microservices antara lain skalabilitas, fleksibilitas, dan dapat dikembangkan oleh beberapa tim yang bekerja secara terpisah.
	\end{frame}
	
	\begin{frame}
		\frametitle{Karakteristik Microservices}
		Lorem ipsum dolor sit amet, consectetur adipiscing elit, sed do eiusmod tempor incididunt ut labore et dolore magna aliqua. Ut enim ad minim veniam, quis nostrud exercitation ullamco laboris nisi ut aliquip ex ea commodo consequat. Duis aute irure dolor in reprehenderit in voluptate velit esse cillum dolore eu fugiat nulla pariatur. Excepteur sint occaecat cupidatat non proident, sunt in culpa qui officia deserunt mollit anim id est laborum.
	\end{frame}
	
	\begin{frame}
		\frametitle{Kelebihan Microservices}
		Berikut adalah beberapa kelebihan dari menggunakan arsitektur microservices dalam pengembangan perangkat lunak:
		\begin{itemize}
			\item Scalability: Arsitektur microservices memungkinkan skalabilitas yang lebih baik dibandingkan dengan monolithic architecture.
			\item Fleksibilitas: Dalam arsitektur microservices, setiap layanan dapat dikembangkan secara terpisah tanpa mempengaruhi layanan lainnya.
			\item Toleransi Kesalahan: Jika terjadi kesalahan pada satu layanan, maka layanan lainnya masih dapat berjalan normal dan tidak terganggu.
			\item Skalabilitas tim: Dalam arsitektur microservices, tim pengembang dapat fokus pada layanan tertentu dan membuat perubahan dengan cepat tanpa harus memikirkan bagaimana perubahan tersebut akan memengaruhi layanan lain dalam aplikasi.
			\item Teknologi yang beragam: Dalam arsitektur microservices, setiap layanan dapat menggunakan teknologi yang berbeda.
			\item Skalabilitas bisnis: Dalam arsitektur microservices, setiap layanan dapat berjalan secara independen.
		\end{itemize}
	\end{frame}
	
	\begin{frame}
		\frametitle{Kekurangan Microservices}
		\begin{itemize}
			\item Kompleksitas: Penggunaan arsitektur microservices dapat meningkatkan kompleksitas sistem secara keseluruhan.
			\item Koordinasi yang lebih rumit: Akibat dari sistem yang menjadi kompleks, koordinasi antar layanan mungkin agak lebih rumit.
			\item Perlu banyak automation: Microservices juga membutuhkan sistem automation yang cukup tinggi untuk bisa melakukan deployment.
			\item Biaya: Penggunaan arsitektur microservices dapat memerlukan biaya yang lebih tinggi.
		\end{itemize}
	\end{frame}
	
	\begin{frame}
		\frametitle{Penerapan Microservices pada aplikasi}
		Penerapan Microservices dalam aplikasi memungkinkan pembagian tugas dan tanggung jawab menjadi lebih terfokus, sehingga dapat memudahkan pengembangan dan pengelolaan aplikasi secara terpisah. Setiap layanan dalam arsitektur Microservices dapat dikembangkan secara independen oleh tim yang berbeda, sehingga proses pengembangan dapat lebih cepat dan efisien. Selain itu, Microservices juga memungkinkan penggunaan teknologi yang berbeda-beda untuk setiap layanan, yang dapat meningkatkan fleksibilitas dan skalabilitas aplikasi.
	\end{frame}
	
	\begin{frame}
		\frametitle{Contoh Penerapan}
		Contoh penerapan Microservices dapat ditemukan pada aplikasi e-commerce seperti Shopee dan Gojek, yang menggunakan banyak layanan terpisah untuk setiap fitur aplikasi seperti pembayaran, pengiriman, dan pemesanan. Dengan menggunakan Microservices, aplikasi dapat diintegrasikan dengan mudah dan dapat berjalan secara independen, sehingga memudahkan dalam pemeliharaan dan pengembangan aplikasi secara keseluruhan.
	\end{frame}
	
\end{document}
